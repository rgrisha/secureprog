
\documentclass{beamer}
\usepackage{polyglossia}
\setdefaultlanguage{english}
\usepackage{textpos} % package for the positioning
\usepackage{graphicx}
\usepackage{tikz}
\usetikzlibrary{chains,fit,shapes,calc,matrix}
%\usepackage{hyperref}

%for drawing memory maps
\input binhex

\usecolortheme[RGB={0,58,108}]{structure}
\usetheme[height=0ex]{Rochester}

\usefonttheme{serif}
\setbeamertemplate{items}[ball] 
\setbeamertemplate{blocks}[rounded][shadow=true] 
\setbeamertemplate{navigation symbols}{} 
\useoutertheme{infolines} 
\useinnertheme{rounded} 
\date{\today} 
\institute{VGTU} 

\definecolor{vgtubg}{RGB}{233,233,233}
\definecolor{vgtufg}{RGB}{0,58,108}
\setbeamercolor{vgtufgbg}{fg=vgtufg,bg=vgtubg}

%\usepackage{tabularx}
\usepackage{listings}
\lstloadlanguages{Java}
\lstset{language=Java}

\setbeamertemplate{headline}{}
\setbeamertemplate{frametitle}{
  \leavevmode%
  %\fbox{%
  \begin{beamercolorbox}[wd=\paperwidth,ht=4ex,dp=0.5ex]{vgtufgbg}
    %%{secsubsec}%
    %%\raggedright

	\hbox{
		\raisebox{-1pt}[0pt][0pt]{ \includegraphics[scale=0.4]{vgtu}}

		\hspace*{2em}%
		{\sffamily\Large\color{vgtufg}\thesection.\thesubsection. \insertframetitle

		%\insertsection
		%\hfill\insertsubsection
		}
	}%
    %\hspace*{2em}%
  \end{beamercolorbox}%
  %}\vskip0pt%
}



\newcommand*{\tn}[2]{\tikz[baseline,remember picture]\node[inner sep=0pt,anchor=base] (#1) {#2};}

\begin{document}
\author{Rolandas Griškevičius} 
\title{SOLID}   

\frame{\titlepage} 

\frame{\frametitle{Definition} 
   A mnemonic acronym, promoted by Robert C. Martin, a.k.a Uncle Bob
   \begin{block}{Single responsibility principle}
     a class should have only a single responsibility (i.e. changes to only one part of the software's specification should be able to affect the specification of the class)
   \end{block}
   \begin{block}{Open/closed principle}
     “software entities … should be open for extension, but closed for modification.”
   \end{block}
   \begin{block}{Liskov substitution principle}
     “objects in a program should be replaceable with instances of their subtypes without altering the correctness of that program.” See also design by contract.
   \end{block}
}

\frame{\frametitle{Definition, cont.} 
   \begin{block}{Interface segregation principle}
     “many client-specific interfaces are better than one general-purpose interface.”
   \end{block}
   \begin{block}{Dependency inversion principle}
     one should “depend upon abstractions, notconcretions.”
   \end{block}
}

\defverbatim[colored]\srpa{%
\begin{lstlisting}[tabsize=8,basicstyle=\ttfamily,basicstyle=\footnotesize,frame=single]
class Person {

  private String name;

  public String getName() {
    return name;
  }

  public void save(BufferedWriter writer) {
    writer.write(getName());
  }
}
\end{lstlisting}
}

\frame{\frametitle{Single responsibility principle} 
What if we need to save person to XML/database/remote API later ?
\srpa
}

\defverbatim[colored]\srpba{%
\begin{lstlisting}[tabsize=8,basicstyle=\ttfamily,basicstyle=\footnotesize,frame=single]
class Person {

  private String firstName;
  private String lastName;

  public String getFirstName() {
    return name;
  }
  public String getLastName() {
    return name;
  }
  
  // No save method anymore
}
\end{lstlisting}
}

\defverbatim[colored]\srpbb{%
\begin{lstlisting}[tabsize=8,basicstyle=\ttfamily,basicstyle=\footnotesize,frame=single]
class PersonFileSaver {
  ...
  BufferedWriter writer;
  public void save(Person person) {
    writer.write(person.getFirstName() + " " + 
      person.getLastName());
  }
}

class PersonXmlSaver {
    ...
  Document document;
  public void save(Person person) {
    Element rootElement = document.createElement("Person");
    document.appendChild(rootElement);
    ....
  }
}
\end{lstlisting}
}



\frame{\frametitle{Single responsibility principle} 
Separate concerns: saving and person data entity
    \srpba
}

\frame{\frametitle{Single responsibility principle} 
Separate concerns: saving and person data entity. Special classes for each medium. Do not save on classes in Java!
    \srpbb
}

\defverbatim[colored]\ocpa{%
\begin{lstlisting}[tabsize=8,basicstyle=\ttfamily,basicstyle=\scriptsize,frame=single]
class Person {

  private String firstName;
  ...
  private String type;
  ...
  public boolean canTeachInUniversity() {
    if("student".equals(type)) return false;
    if("professor".equals(type)) return true;
    return false;
  }
}

class Student extends Person ....
class Professor extends Person ....

\end{lstlisting}
}

\frame{\frametitle{Open Closed principle} 
This class is not closed for modification. What if student, but being a master is allowed to teach in University?
Modifications if logic might be endless.
\ocpa
}


\defverbatim[colored]\ocpb{%
\begin{lstlisting}[tabsize=8,basicstyle=\ttfamily,basicstyle=\scriptsize,frame=single]
class Person {
  ...
  public abstract boolean canTeachInUniversity();
}

class Student extends Person {
  @Override
  public abstract boolean canTeachInUniversity() {
    if(isMaster()) return true;
    return false;
  }
}

\end{lstlisting}
}

\frame{\frametitle{Open Closed principle} 
Now any type of person can express itself - can this type of person teach in University.
\ocpb
}


\defverbatim[colored]\lspa{%
\begin{lstlisting}[tabsize=8,basicstyle=\ttfamily,basicstyle=\footnotesize,frame=single]
class Person {

  private String firstName;
  private String lastName;
  private String emailAddress;

  public String getFirstName() {
    return name;
  }
  ...
  public String getEmailAddress() {
    return emailAddress;
  }

}

\end{lstlisting}
}

\defverbatim[colored]\lspb{%
\begin{lstlisting}[tabsize=8,basicstyle=\ttfamily,basicstyle=\footnotesize,frame=single]

  class NotVerifiedPerson extends Person {

    /* We break contract here. We can use UnverifiedPerson 
       instead of Person because of inheritance, but it will 
       throw exception unexpectedly, but Person does not 
       throw exceptions!!! */
    @Override
    public String getEmailAddress() {
      throw new RuntimeException(
        "Person email address is not verified");
      return null;
    }

  }
}

\end{lstlisting}
}


\frame{\frametitle{Liskov substitution principle} 
Functions that use references to base classes must be able to use objects of the derived class without knowing it
We have initial class of Person:
\lspa
}

\frame{\frametitle{Liskov substitution principle} 
Functions that use references to base classes must be able to use objects of the derived class without knowing it
\\Say we care a lot about verification and unverified person will gain no access in lot of places in our system
\lspb
}

\defverbatim[colored]\lspc{%
\begin{lstlisting}[tabsize=8,basicstyle=\ttfamily,basicstyle=\scriptsize,frame=single]

  interface VerifiedEmail {
    String getVerifiedEmailAddress();
  }

  interface UnverifiedEmail {
    boolean verifyEmailAddress() {}
  }

  class VerifiedPerson extends Person implements VerifiedEmail {
    
  }

  class UnverifiedPerson extends Person implements UnverifiedEmail {
    EmailVerificator verificator;
    ....
    public void verifyEmailAddress() {
      return verificator.verify(getEmailAdrress());
    }
  }
}

\end{lstlisting}
}

\frame{\frametitle{Liskov substitution principle} 
We need clearly split between verified and unverified person here.
\textbf{Note interfaces.} They help align types
\lspc
}


\defverbatim[colored]\dipa{%
\begin{lstlisting}[tabsize=8,basicstyle=\ttfamily,basicstyle=\scriptsize,frame=single]

  class Person {
    ...
    String id;
    ...
    public boolean teachesInUniversity() {
      University university = new University();
      return university.isTeacher(id);
    }
  }
}

\end{lstlisting}
}

\frame{\frametitle{Dependency inversion principle} 
Two bad things here:
\begin{itemize}
  \item University object is tight coupled. Problems with unit tests.
  \item Depends on sigle University type. What if we sell our sofware to governmental institution that controls some type of schools, not only universities?
\end{itemize}
\dipa
}

\defverbatim[colored]\dipb{%
\begin{lstlisting}[tabsize=8,basicstyle=\ttfamily,basicstyle=\scriptsize,frame=single]

  class Person {
    ...
    String id;
    University university;
    ...
    public void setUniversity(University u) {
      university = u;
    }

    public boolean teachesInUniversity() {
      return university.isTeacher(id);
    }
  }
}

\end{lstlisting}
}

\frame{\frametitle{Dependency inversion principle} 
First item solved. We do not depend on University instance, but still depend university type.
\\Since we inject type, it could be more abstract
\dipb
}

\defverbatim[colored]\dipc{%
\begin{lstlisting}[tabsize=8,basicstyle=\ttfamily,basicstyle=\tiny,frame=single]

  interface EducationInstitution() {
    boolean isTeacher(Person p);
  }

  class University implements EducationInstitution ...

  class Person {
    ...
    String id;
    EducationInstitution educationInstitution ;
    ...
    public void setInstitution(EducationInstitution ei) {
      educationInstitution = ei;
    }

    public boolean teachesInUniversity() {
      return educationInstitution.isTeacher(id);
    }
  }

  // in some method somewhere
  .....
  EducationInstitution ei = new University();
  person.isTeacher(ei);
  .....
  EducationInstitution ei = new School();
  person.isTeacher(ei);


\end{lstlisting}
}

\frame{\frametitle{Dependency inversion principle} 
Now we are not dependent on neither type nor instance.
\dipc
}




\end{document}

